\documentclass[12pt]{beamer}
%\usepackage[margin=1.25in]{geometry}                % See geometry.pdf to learn the layout 
\usepackage{graphicx}
\usepackage{amssymb}
\usepackage{epstopdf}
\usepackage{hyperref}
\usepackage{setspace}
\usepackage{natbib}
\usepackage{booktabs}
\usepackage{amsmath}
\usepackage{caption}

\begin{document}
\title{Using Machine Learning to Infer Unobserved Firm Networks---Part 2: benchmarking}
\author{Jesse Tweedle}
%\date{}

\begin{frame}{}

\maketitle

\end{frame}

\begin{frame}{Outline}
\begin{enumerate}
\item Problem
\item Needs
\item Approach: separate data, benchmarking, methods steps, improve each one separately.
\item Data sources
\item Methods
\item Results
\end{enumerate}
\end{frame}

\begin{frame}{Problem}

\begin{block}{Goal: study firm-firm trade}
To understand
\begin{itemize}
\item supply chains and vertical integration
\item intra-firm trade
\item and more
\end{itemize}
\end{block}

\begin{block}{Problem: we don't have firm-firm transaction data}
But: 
\begin{itemize}
\item we have lots of useful data
\item and an idea of how to use it
\end{itemize}
\end{block}

%Have lots of data. Want to infer firm-firm trading relationships. But: want it to be internally consistent---a firm can't produce \$10 million of output and only have customers that buy \$1,000. Also need it to be externally consistent: if you add up all the firm-firm relationships, you should get back to industry accounts. More detail on this later.

\end{frame}

\begin{frame}{Facts / needs}
% this slide: some things: few relationships. 
\begin{block}{Facts: start with manufacturing}
\begin{itemize}
\item $\approx 30$k plants. 
\item $\approx 900$m possible connections.
\item $\approx 70$ industries (IOIC)
\item $x$ goods (detailed confidential IOCC)
\end{itemize}
\end{block}

\begin{block}{Needs}
\begin{itemize}
\item Data that indicates relationship between plants
\item Method to identify relationships
\item Method to benchmark to make sure it all adds up
\end{itemize}
\end{block}

\end{frame}



\begin{frame}{Normal approach: idea}

\begin{block}{To achieve this:}
\begin{itemize}
\item Data that indicates relationship between plants
\item Method to identify relationships
\end{itemize}
\end{block}

\begin{block}{Do this:}
\begin{itemize}
\item Use supply-use/input-output tables
\item Assume every firm-firm relationship is the same as the industry IO relationship
\end{itemize}
\end{block}

\end{frame}

\begin{frame}{Normal approach: problems}

% use full thing here, or just manufacturing?
\begin{block}{Implies way too many plant-plant relationships}
\begin{itemize}
\item Use goods-only, square IO table, DC level
\item $\approx 70^2 = 4900$ possible connections
\item $\approx $ actual connections
\item 50\% density---half of the possible connections are given
\item Gets worse using full table: 90\% density of $50000+$ possible connections
\end{itemize}
\end{block}

% HERE: why is this wrong.

% every plant in an industry trades with every other plant in the other industry (with positive entry in IO table)
% if industries are the same, then plant trades with itself---wrong

% To get an idea of better number. Just use ASM commodity files. Commodities just don't line up.
% Can't just cut IO table off at some direct requirement level either, because you drop so much expenditure.


\end{frame}

\begin{frame}{Normal approach: problems}

% assign small plant as important supplier to big plant---IO numbers are not consistent with plant numbers
% may be missing some connections? industry definitions or something.
% so still need to benchmark.

\begin{block}{Plant and IO data may not be consistent}
\begin{itemize}
\item Relationship may not be consistent with plant data % may be assigning tiny auto parts plant as an important supplier to GM assembly plant
\item Plant-plant relationships may not be consistent with industry IO
\end{itemize}
\end{block}
\end{frame}


\begin{frame}{Needs}

Data: firm-firm (really location-location or establishment-establishment, we hope). We'll take anything that indicates a relationship between establishments. Method to infer links between establishments based on those data.  Method to benchmark to the national accounts, and itself. Then look at results. Also want to seperate these things. 

\end{frame}

\begin{frame}{Approach}
(1) Start with manufacturing. Get data + methods to work. Check results. Refine each step.  

Data: STF, IO, IPTF, ASM, etc.

Method: pick possible links using the data, then Lasso to benchmark / solve system of equations.

\end{frame}

\begin{frame}{Data sources (for now)}

Data: STF, IO, IPTF, ASM, etc. Describe each one.

\end{frame}

\begin{frame}{Methods}

(1) Use IO, STF, ASM, IPTF to give any possible link between firms (e.g., an upper bound on links between firms), then a subset that we think is most likely (a lower bound on the set of links between firms).

(2) Benchmark to make expenditures between firms internally and externally consistent. Need to solve a huge, underdetermined system of equations, a big linear programming problem (tens-of-thousands of equations, hundreds of millions of parameters, maybe). Lasso is a good way.

\end{frame}

\begin{frame}{Results}

(1) it works, doesn't crash the server (for now).  (2) it's relatively fast (x mins to solve manufacturing problem. (3) it works, relatively well, needs more refinement in input data to get it to work better---import/export registry, IO tables, final demand and such.

\end{frame}

%\bibliographystyle{chicago}
%\bibliography{inferring-firm-networks-2}

\end{document}








